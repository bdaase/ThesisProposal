\section{Related Work}
\label{sec:related-work}

Both \acp{spe}, as well as column stores, are well-researched topics.
However, their combination has so far been limited to their integration by MonetDB/Datacell~\cite{DBLP:journals/pvldb/LiarouIMK12} and Trill~\cite{DBLP:journals/pvldb/ChandramouliGBDPTW14} without a detailed analysis of the advantages and disadvantages of such an integration.

\subsection{Stream Processing Egines}

A repeatedly mentioned difference between modern \acp{spe} is the underlying processing model~\cite{DBLP:journals/pvldb/ZeuchBRMKLRTM19}, which can be either micro-batching or pipeline-tuple-at-a-time processing~\cite{DBLP:journals/pvldb/AkidauBCCFLMMPS15,DBLP:conf/cloud/BattreEHKMW10,DBLP:conf/pldi/ChambersRPAHBW10,DBLP:conf/osdi/DeanG04,DBLP:conf/nsdi/ZahariaCDDMMFSS12}.
Micro-batching \acp{spe} divide data streams into batches, i.e. finite sets of tuples, and process one batch at a time.
Prominent examples of that execution model are Spark~\cite{DBLP:journals/cacm/ZahariaXWDADMRV16} and SABER~\cite{DBLP:conf/sigmod/KoliousisWFWCP16}.
In contrast, tuple-at-a-time systems, have long-standing data-parallel pipelines that consist of stream transformations, where instead of splitting streams into micro-batches, operators retrieve individual tuples and continuously generate output tuples.
\acp{spe} such as Apache Flink~\cite{DBLP:journals/cacm/ZahariaXWDADMRV16} and Storm~\cite{DBLP:conf/sigmod/ToshniwalTSRPKJGFDBMR14} use this approach.
However, usually left out of this discussion is the fact that in both models, the data already arrives in larger buffers from other systems.
Such a buffer, which we call \emph{nano-batch}, has at least the size of a network buffer but can be of any size depending on the concrete application.
Thus, regardless of the specific execution model, it is possible to perform batched processing even in tuple-at-a-time systems.

Additionally, in the last decades \acp{spe} have been developed in two different directions: As scale-out as well as scale-up systems.
Scale-out systems are optimized for executing streaming queries on distributed, heterogenous, shared-nothing architectures and thus are of limited relevance to our analyses. 
Of much more interest to us are scale-up systems, which are optimized for execution on a single machine to efficiently utilize the capabilities of modern high-end systems.
Current \acp{spe} in this category include Streambox~\cite{DBLP:conf/usenix/MiaoPJPML17}, Trill~\cite{DBLP:journals/pvldb/ChandramouliGBDPTW14}, and SABER.
Streambox in particular aims to optimize execution for multicore machines, while SABER considers heterogeneous processing through the use of GPUs.
Finally, Trill scores high on flexibility, supporting a wide range of queries with SQL as well as streaming queries.
However, all of these systems focus primarily on optimizing computational performance, but not on optimizing memory patterns and access.

\subsection{Column Stores}

The MonetDB systems~\cite{DBLP:conf/cidr/BonczZN05,DBLP:journals/vldb/BonczK99} pioneered the development of modern column-oriented database systems and vectorized query execution.
As a result, a wide range of columnar database systems was created to follow this approach~\cite{DBLP:conf/sigmod/SikkaFLCPB12,DBLP:conf/vldb/StonebrakerABCCFLLMOORTZ05,DBLP:conf/icde/ZukowskiWB12,DBLP:journals/pvldb/GrundKPZCM10}.
\citet{DBLP:journals/pvldb/LiarouIMK12} already showed that column-oriented designs can significantly outperform row-based databases in standardized benchmarks such as TPC-H due to superior CPU and cache performance~\cite{DBLP:conf/sigmod/AbadiMH08}.

The work of \citet{DBLP:conf/sigmod/SikkaFLCPB12} is thereby of particular interest.
While previous work has highlighted the advantages of column stores compared to row stores for OLAP queries, the authors show that column-oriented systems can also outperform row-oriented ones for OLTP queries.
For increasingly complex OLTP queries with even newly designed and more complex benchmarks like TPC-E~\cite{DBLP:conf/sigmod/SikkaFLCPB12,DBLP:conf/edbt/TozunPKJA13,tpce}, the nano-batch sizes in the streaming could already contain enough tuples so that column-oriented streaming deliver better performance than row-oriented streaming.

A detailed comparison of how column-oriented and row-oriented databases compare to each other was given by \citet{DBLP:conf/sigmod/AbadiMH08}, however, showing that simulating a row store in a column store does not yield good results.
While vectorization of specific parts of the code can improve performance dramatically~\cite{DBLP:journals/pvldb/KerstenLKNPB18}, \citet{DBLP:conf/vldb/HarizopoulosLAM06} compare the performance of a row and column store built from scratch and demonstrates that in a carefully controlled environment, column stores still outperform row stores.