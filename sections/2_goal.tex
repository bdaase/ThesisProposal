\section{Goal of Thesis}
\label{sec:goal}

In order to understand the impact of a column-based compared to a row-based layout for an \acp{spe}, we analyze their raw query performance along several microbenchmarking dimensions.
Therefore, we store the data as an \ac{aos} when simulating a row-store and as an \ac{soa} in the case of a column store and analyze their performance across an exhausting set of queries.
Independently of this, we will look at the transformation of incoming tuples into a columnar-based store as well as their materialization back into a row-based format at the end of the query.
This also raises the question of whether hybrid forms of data storage are worthwhile.
If, for example, some fields are not used during the actual processing (i.e. no filters, joins or aggregations take place on them), but they have to be finally issued with the tuples as payload, this cold data could be kept in a row-store while the hot data is processed in a column-store.

An additional goal will be to also understand how these different approaches perform on different hardware.
As shown by~\citet{DBLP:journals/pvldb/KerstenLKNPB18}, vectorizing plays a central role in efficient query processing.
However, with different vectorizing extensions on different platforms as well as different kinds of memory on different machines~\cite[]{bollmeier2021processor}, especially the aforementioned \ac{hbm}, the performance of row-store-based and column-store-based streaming could be influenced significantly.
For reference, the \ac{hbm} ARM systems can reach up to 3x sequential memory bandwidth compared to state-of-the-art Intel, AMD, and Power systems which peak around 200 GB/s~\cite[]{bollmeier2021processor}.

As a  result, we can hopefully create a cost model that decides at query compilation time whether it will process data more efficiently row-based, column-based, or hybrid.
Finally, we test the suitability of our cost model and non-row-store streaming in the Darwin stream processing engine~\cite[]{DBLP:conf/cidr/BensonR22}.

Given that, we plan the following contributions:
\begin{enumerate}
    \item We understand the impact of different data layouts (row-based, column-based, or hybrid) on the query performance of \acp{spe}.
    \item We create a cost model that allows to pick the best layout at query compilation time and validate that in the Darwin stream processing engine.
\end{enumerate}