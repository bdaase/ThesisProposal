\section{Goal of Thesis}
\label{sec:goal}

In order to understand the impact of a column-based compared to a row-based layout for an \ac{spe}, we analyze their raw query performance along several microbenchmarking dimensions.
Therefore, we select a set of Nexmark queries~\cite{tucker2008nexmark} to be able to pinpoint the influence on query performance depending on the relational algebra operators.
We will further investigate a few side branches to also understand the influence of preceding input parsing, explicit SIMD operations, the underlying memory architecture and vectorizing extension, different window types, and multithreading. 
Independently of the benefits, we will look at the transformation of incoming tuples into a columnar-based store as well as their materialization back into a row-based format at the end of the query so that we can estimate the costs.
As a result, this allows us to select a suitable storage format before query execution.

After analyzing the design space in several microbenchmarks, we test the suitability of non-row store storage in the Darwin stream processing engine~\cite{DBLP:conf/cidr/BensonR22}.
At this point, we get even more concrete application scenarios and can further tune our hybrid approach. 
Given that, we plan the following contributions:
\begin{enumerate}
    \item We understand the impact of different data layouts (row-based, column-based, or hybrid) on individual relational algebra operators.
    \item We relate the benefits and costs of column-based processing to each other, allowing the selection of a suitable format at query compilation time.
    \item We validate our approach in the Darwin stream processing engine~\cite[]{DBLP:conf/cidr/BensonR22}.
\end{enumerate}